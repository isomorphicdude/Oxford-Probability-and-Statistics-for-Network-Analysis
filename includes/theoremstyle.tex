% modify innerrightmargin if floats were lost

\usepackage{amsfonts, amsmath, amssymb, amsthm, thmtools, bm}
\usepackage{avant} % Use the Avantgarde font for headings
\usepackage[most]{tcolorbox}

% Boxed/framed environments
\newtheoremstyle{royalnumbox}%
{0pt}% Space above
{0pt}% Space below
{\normalfont}% Body font
{}% Indent amount
{\small\bf}% Theorem head font
{\;}% Punctuation after theorem head
{0.25em}% Space after theorem head
{   \thmname{#1} 
    \thmnumber{#2} \thmnote{\bfseries\color{black}---\nobreakspace#3.}} % Optional theorem note
% \renewcommand{\qedsymbol}{$\blacksquare$}% Optional qed square

\newtheoremstyle{blacknumex}% Theorem style name
{5pt}% Space above
{5pt}% Space below
{\normalfont}% Body font
{} % Indent amount
{\small\bf}% Theorem head font
{\;}% Punctuation after theorem head
{0.25em}% Space after theorem head
{
    \thmname{#1}
    \thmnumber{#2}
    \thmnote{---\nobreakspace#3.}}% Optional theorem note

\newtheorem*{notation}{Notation}
\newtheorem*{hint}{Hint}
\newtheorem*{solution}{Solution}

\newcounter{dummy} 
\numberwithin{dummy}{section}

\theoremstyle{royalnumbox}
\newtheorem{definitionT}[dummy]{Definition}
\newtheorem{theoremT}[dummy]{Theorem}
\newtheorem{lemmaT}[dummy]{Lemma}
\newtheorem{corollaryT}[dummy]{Corollary}
\newtheorem{propositionT}[dummy]{Proposition}
\newtheorem{propertyT}[dummy]{Property}
\newtheorem{remarkT}[dummy]{Remark}

\theoremstyle{blacknumex}
\newtheorem{exampleT}[dummy]{Example}
\newtheorem{exerciseT}[dummy]{Exercise}

\numberwithin{equation}{section}

\RequirePackage[framemethod=TikZ]{mdframed}

\newcounter{definition}

% % Definition box
% \newtcolorbox{dBox}{
%   enhanced,
%   breakable,
%   arc=5pt, outer arc=5pt,
%   colback=reddish!10, 
%   colframe=reddish,
%   boxrule=1pt,
%   left=5pt, 
%   right=5pt, 
%   top=5pt, 
%   bottom=5pt,
% %   skipabove=7pt, skipbelow=7pt
% }



% % Main Theorem box
% \newtcolorbox{tBox}{
%   enhanced,
%   breakable,
%   arc=5pt, 
%   outer arc=5pt,
%   colback=c0!10, 
%   colframe=c0!10,
%   boxrule=1pt,
%   left=5pt, 
%   right=5pt, 
%   top=5pt, 
%   bottom=5pt,
% %   skipabove=7pt, skipbelow=7pt
% }


% % Lemma/Corollary/Proposition/Property box
% \newtcolorbox{lBox}{
%   enhanced,
%   breakable,
%   arc=5pt,
%   outer arc=5pt,
%   colback=c0!10,
%   colframe=c0!10,
%   boxrule=1pt,
%   left=5pt,
%   right=5pt,
%   top=5pt,
%   bottom=5pt,
% %   skipabove=7pt, skipbelow=7pt
% }

% % Example/Remark/Exercise box
% \newtcolorbox{exBox}{
%   enhanced,
%   breakable,
%   arc=5pt, 
%   outer arc=5pt,
%   colback=mossgreen!10!white,
%   colframe=mossgreen,
%   boxrule=1pt,
%   left=5pt,
%   right=5pt,
%   top=5pt,
%   bottom=5pt,
% %   skipabove=7pt, skipbelow=7pt
% }

% % Extra content box for contents not covered in the lecture notes
% \newtcolorbox{unexamBox}{
%   enhanced,
%   breakable,
%   arc=20pt, 
%   outer arc=20pt,
%   colback=gray!10,
%   colframe=gray,
%   boxrule=1pt,
%   left=5pt, 
%   right=5pt, 
%   top=5pt, 
%   bottom=5pt,
% %   skipabove=7pt, skipbelow=7pt
% }

% Definition box
\newtcolorbox{dBox}{
  enhanced,
  breakable,
  arc=0pt,
  outer arc=0pt,
  colback=skyblue!10,  % Light blue
  colframe=skyblue!10,    
  boxrule=0pt,
  left=5pt, 
  right=5pt, 
  top=5pt, 
  bottom=5pt,
}

% Main Theorem box
\newtcolorbox{tBox}{
  enhanced,
  breakable,
  arc=0pt, 
  outer arc=0pt,
  colback=yellow!10,   % Light yellow
  colframe=yellow!10,
  boxrule=0pt,
  left=5pt, 
  right=5pt, 
  top=5pt, 
  bottom=5pt,
}

% Lemma/Corollary/Proposition/Property box
\newtcolorbox{lBox}{
  enhanced,
  breakable,
  arc=0pt,
  outer arc=0pt,
  colback=orange!10,   % Light orange
  colframe=orange!10,
  boxrule=0pt,
  left=5pt,
  right=5pt,
  top=5pt,
  bottom=5pt,
}

% Example/Remark/Exercise box
\newtcolorbox{exBox}{
  enhanced,
  breakable,
  arc=0pt, 
  outer arc=0pt,
  colback=green!5!white,   % Light green background
  colframe=green!10!white,  % Slightly darker green for the frame
  boxrule=0pt,
  left=5pt,
  right=5pt,
  top=5pt,
  bottom=5pt,
}


% Extra content box for contents not covered in the lecture notes
\newtcolorbox{unexamBox}{
  enhanced,
  breakable,
  arc=0pt, 
  outer arc=0pt,
  colback=gray!10,    % Light gray
  colframe=gray!10,
  boxrule=0pt,
  left=5pt, 
  right=5pt, 
  top=5pt, 
  bottom=5pt,
}

\newenvironment{unexaminable}{\begin{unexamBox}}{\end{unexamBox}}




% Creates an environment for each type of theorem and assigns it a theorem text style from the "Theorem Styles" section above and a colored box from above
\newenvironment{definition}{\begin{dBox}\begin{definitionT}}{\end{definitionT}\end{dBox}}

\newenvironment{theorem}{\begin{tBox}\begin{theoremT}}{\end{theoremT}\end{tBox}}

\newenvironment{lemma}{\begin{lBox}\begin{lemmaT}}{\end{lemmaT}\end{lBox}}

\newenvironment{proposition}{\begin{lBox}\begin{propositionT}}{\end{propositionT}\end{lBox}}

\newenvironment{corollary}{\begin{lBox}\begin{corollaryT}}{\end{corollaryT}\end{lBox}}

\newenvironment{property}{\begin{lBox}\begin{propertyT}}{\end{propertyT}\end{lBox}}


\newenvironment{remark}{\begin{exBox}\begin{remarkT}}{\end{remarkT}\end{exBox}}

\newenvironment{example}{\begin{exBox}\begin{exampleT}}{{}\end{exampleT}\end{exBox}}